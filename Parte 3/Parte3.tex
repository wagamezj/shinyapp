\documentclass[]{article}
\usepackage{lmodern}
\usepackage{amssymb,amsmath}
\usepackage{ifxetex,ifluatex}
\usepackage{fixltx2e} % provides \textsubscript
\ifnum 0\ifxetex 1\fi\ifluatex 1\fi=0 % if pdftex
  \usepackage[T1]{fontenc}
  \usepackage[utf8]{inputenc}
\else % if luatex or xelatex
  \ifxetex
    \usepackage{mathspec}
  \else
    \usepackage{fontspec}
  \fi
  \defaultfontfeatures{Ligatures=TeX,Scale=MatchLowercase}
\fi
% use upquote if available, for straight quotes in verbatim environments
\IfFileExists{upquote.sty}{\usepackage{upquote}}{}
% use microtype if available
\IfFileExists{microtype.sty}{%
\usepackage{microtype}
\UseMicrotypeSet[protrusion]{basicmath} % disable protrusion for tt fonts
}{}
\usepackage[margin=1in]{geometry}
\usepackage{hyperref}
\hypersetup{unicode=true,
            pdftitle={Parte 3 Resultados y explicacion},
            pdfauthor={Willmer},
            pdfborder={0 0 0},
            breaklinks=true}
\urlstyle{same}  % don't use monospace font for urls
\usepackage{color}
\usepackage{fancyvrb}
\newcommand{\VerbBar}{|}
\newcommand{\VERB}{\Verb[commandchars=\\\{\}]}
\DefineVerbatimEnvironment{Highlighting}{Verbatim}{commandchars=\\\{\}}
% Add ',fontsize=\small' for more characters per line
\usepackage{framed}
\definecolor{shadecolor}{RGB}{248,248,248}
\newenvironment{Shaded}{\begin{snugshade}}{\end{snugshade}}
\newcommand{\AlertTok}[1]{\textcolor[rgb]{0.94,0.16,0.16}{#1}}
\newcommand{\AnnotationTok}[1]{\textcolor[rgb]{0.56,0.35,0.01}{\textbf{\textit{#1}}}}
\newcommand{\AttributeTok}[1]{\textcolor[rgb]{0.77,0.63,0.00}{#1}}
\newcommand{\BaseNTok}[1]{\textcolor[rgb]{0.00,0.00,0.81}{#1}}
\newcommand{\BuiltInTok}[1]{#1}
\newcommand{\CharTok}[1]{\textcolor[rgb]{0.31,0.60,0.02}{#1}}
\newcommand{\CommentTok}[1]{\textcolor[rgb]{0.56,0.35,0.01}{\textit{#1}}}
\newcommand{\CommentVarTok}[1]{\textcolor[rgb]{0.56,0.35,0.01}{\textbf{\textit{#1}}}}
\newcommand{\ConstantTok}[1]{\textcolor[rgb]{0.00,0.00,0.00}{#1}}
\newcommand{\ControlFlowTok}[1]{\textcolor[rgb]{0.13,0.29,0.53}{\textbf{#1}}}
\newcommand{\DataTypeTok}[1]{\textcolor[rgb]{0.13,0.29,0.53}{#1}}
\newcommand{\DecValTok}[1]{\textcolor[rgb]{0.00,0.00,0.81}{#1}}
\newcommand{\DocumentationTok}[1]{\textcolor[rgb]{0.56,0.35,0.01}{\textbf{\textit{#1}}}}
\newcommand{\ErrorTok}[1]{\textcolor[rgb]{0.64,0.00,0.00}{\textbf{#1}}}
\newcommand{\ExtensionTok}[1]{#1}
\newcommand{\FloatTok}[1]{\textcolor[rgb]{0.00,0.00,0.81}{#1}}
\newcommand{\FunctionTok}[1]{\textcolor[rgb]{0.00,0.00,0.00}{#1}}
\newcommand{\ImportTok}[1]{#1}
\newcommand{\InformationTok}[1]{\textcolor[rgb]{0.56,0.35,0.01}{\textbf{\textit{#1}}}}
\newcommand{\KeywordTok}[1]{\textcolor[rgb]{0.13,0.29,0.53}{\textbf{#1}}}
\newcommand{\NormalTok}[1]{#1}
\newcommand{\OperatorTok}[1]{\textcolor[rgb]{0.81,0.36,0.00}{\textbf{#1}}}
\newcommand{\OtherTok}[1]{\textcolor[rgb]{0.56,0.35,0.01}{#1}}
\newcommand{\PreprocessorTok}[1]{\textcolor[rgb]{0.56,0.35,0.01}{\textit{#1}}}
\newcommand{\RegionMarkerTok}[1]{#1}
\newcommand{\SpecialCharTok}[1]{\textcolor[rgb]{0.00,0.00,0.00}{#1}}
\newcommand{\SpecialStringTok}[1]{\textcolor[rgb]{0.31,0.60,0.02}{#1}}
\newcommand{\StringTok}[1]{\textcolor[rgb]{0.31,0.60,0.02}{#1}}
\newcommand{\VariableTok}[1]{\textcolor[rgb]{0.00,0.00,0.00}{#1}}
\newcommand{\VerbatimStringTok}[1]{\textcolor[rgb]{0.31,0.60,0.02}{#1}}
\newcommand{\WarningTok}[1]{\textcolor[rgb]{0.56,0.35,0.01}{\textbf{\textit{#1}}}}
\usepackage{graphicx,grffile}
\makeatletter
\def\maxwidth{\ifdim\Gin@nat@width>\linewidth\linewidth\else\Gin@nat@width\fi}
\def\maxheight{\ifdim\Gin@nat@height>\textheight\textheight\else\Gin@nat@height\fi}
\makeatother
% Scale images if necessary, so that they will not overflow the page
% margins by default, and it is still possible to overwrite the defaults
% using explicit options in \includegraphics[width, height, ...]{}
\setkeys{Gin}{width=\maxwidth,height=\maxheight,keepaspectratio}
\IfFileExists{parskip.sty}{%
\usepackage{parskip}
}{% else
\setlength{\parindent}{0pt}
\setlength{\parskip}{6pt plus 2pt minus 1pt}
}
\setlength{\emergencystretch}{3em}  % prevent overfull lines
\providecommand{\tightlist}{%
  \setlength{\itemsep}{0pt}\setlength{\parskip}{0pt}}
\setcounter{secnumdepth}{0}
% Redefines (sub)paragraphs to behave more like sections
\ifx\paragraph\undefined\else
\let\oldparagraph\paragraph
\renewcommand{\paragraph}[1]{\oldparagraph{#1}\mbox{}}
\fi
\ifx\subparagraph\undefined\else
\let\oldsubparagraph\subparagraph
\renewcommand{\subparagraph}[1]{\oldsubparagraph{#1}\mbox{}}
\fi

%%% Use protect on footnotes to avoid problems with footnotes in titles
\let\rmarkdownfootnote\footnote%
\def\footnote{\protect\rmarkdownfootnote}

%%% Change title format to be more compact
\usepackage{titling}

% Create subtitle command for use in maketitle
\providecommand{\subtitle}[1]{
  \posttitle{
    \begin{center}\large#1\end{center}
    }
}

\setlength{\droptitle}{-2em}

  \title{Parte 3 Resultados y explicacion}
    \pretitle{\vspace{\droptitle}\centering\huge}
  \posttitle{\par}
    \author{Willmer}
    \preauthor{\centering\large\emph}
  \postauthor{\par}
      \predate{\centering\large\emph}
  \postdate{\par}
    \date{25/10/2020}


\begin{document}
\maketitle

` \#\# Parte 3

Para esta parte utilizamos principalmente la libreria (dplyr) que
permite de tablas ejecutar facilmente procedimientos presentamos cada
punto en el siguiente archivo:

Subtotalice el ingreso por empresa donde la moneda origen sea dólares

\begin{Shaded}
\begin{Highlighting}[]
\KeywordTok{library}\NormalTok{(readr)}
\KeywordTok{library}\NormalTok{(dplyr)}
\end{Highlighting}
\end{Shaded}

\begin{verbatim}
## Warning: package 'dplyr' was built under R version 3.6.3
\end{verbatim}

\begin{verbatim}
## 
## Attaching package: 'dplyr'
\end{verbatim}

\begin{verbatim}
## The following objects are masked from 'package:stats':
## 
##     filter, lag
\end{verbatim}

\begin{verbatim}
## The following objects are masked from 'package:base':
## 
##     intersect, setdiff, setequal, union
\end{verbatim}

\begin{Shaded}
\begin{Highlighting}[]
\KeywordTok{library}\NormalTok{(stringr)}
\KeywordTok{library}\NormalTok{(zoo)}
\end{Highlighting}
\end{Shaded}

\begin{verbatim}
## 
## Attaching package: 'zoo'
\end{verbatim}

\begin{verbatim}
## The following objects are masked from 'package:base':
## 
##     as.Date, as.Date.numeric
\end{verbatim}

\begin{Shaded}
\begin{Highlighting}[]
\NormalTok{BASE <-}\StringTok{ }\KeywordTok{read_delim}\NormalTok{(}\StringTok{"BASE.txt"}\NormalTok{, }
                   \StringTok{";"}\NormalTok{, }\DataTypeTok{escape_double =} \OtherTok{FALSE}\NormalTok{, }\DataTypeTok{col_types =} \KeywordTok{cols}\NormalTok{(}\DataTypeTok{Fecha =} \KeywordTok{col_date}\NormalTok{(}\DataTypeTok{format =} \StringTok{"%d/%m/%Y"}\NormalTok{)), }
                   \DataTypeTok{trim_ws =} \OtherTok{TRUE}\NormalTok{)}
\NormalTok{subtotal_USD <-}\StringTok{ }\NormalTok{BASE[BASE}\OperatorTok{$}\NormalTok{Moneda }\OperatorTok{==}\StringTok{ 'USD'}\NormalTok{,] }\OperatorTok\StringTok{ }\KeywordTok{select}\NormalTok{(Empresa,Moneda,Ingreso) }\OperatorTok\StringTok{ }\KeywordTok{group_by}\NormalTok{(Empresa,Moneda) }\OperatorTok\StringTok{ }\KeywordTok{summarise}\NormalTok{(}\DataTypeTok{total =} \KeywordTok{sum}\NormalTok{(Ingreso))}
\end{Highlighting}
\end{Shaded}

\begin{verbatim}
## `summarise()` regrouping output by 'Empresa' (override with `.groups` argument)
\end{verbatim}

\begin{Shaded}
\begin{Highlighting}[]
\NormalTok{subtotal_USD}
\end{Highlighting}
\end{Shaded}

\begin{verbatim}
## # A tibble: 3 x 3
## # Groups:   Empresa [3]
##   Empresa   Moneda total
##   <chr>     <chr>  <dbl>
## 1 Empesa B  USD    15.7 
## 2 Empresa A USD     9.86
## 3 Empresa C USD     6.22
\end{verbatim}

Muestre el ingreso total por cada moneda de la empresa A (en moneda
origen)

\begin{Shaded}
\begin{Highlighting}[]
\NormalTok{total_empresaa <-}\StringTok{ }\NormalTok{BASE[BASE}\OperatorTok{$}\NormalTok{Empresa }\OperatorTok{==}\StringTok{ 'Empresa A'}\NormalTok{,] }\OperatorTok\StringTok{ }\KeywordTok{select}\NormalTok{(Empresa,Moneda,Ingreso) }\OperatorTok\StringTok{ }\KeywordTok{group_by}\NormalTok{(Empresa,Moneda) }\OperatorTok\StringTok{ }\KeywordTok{summarise}\NormalTok{(}\DataTypeTok{total =} \KeywordTok{sum}\NormalTok{(Ingreso))}
\end{Highlighting}
\end{Shaded}

\begin{verbatim}
## `summarise()` regrouping output by 'Empresa' (override with `.groups` argument)
\end{verbatim}

\begin{Shaded}
\begin{Highlighting}[]
\NormalTok{total_empresaa}
\end{Highlighting}
\end{Shaded}

\begin{verbatim}
## # A tibble: 3 x 3
## # Groups:   Empresa [1]
##   Empresa   Moneda     total
##   <chr>     <chr>      <dbl>
## 1 Empresa A EURO       0.575
## 2 Empresa A PESO   71668.   
## 3 Empresa A USD        9.86
\end{verbatim}

Cuál de las empresas fue la que menor ingreso en PESOS obtuvo durante el
primer semestre (01/01/2017-30/06/2017)

\begin{Shaded}
\begin{Highlighting}[]
\NormalTok{BASE}\OperatorTok{$}\NormalTok{Empresa =}\StringTok{ }\KeywordTok{str_replace_all}\NormalTok{(BASE}\OperatorTok{$}\NormalTok{Empresa ,}\StringTok{"EmpresaC"}\NormalTok{,}\StringTok{"Empresa C"}\NormalTok{)}
\NormalTok{menor_ingreso <-}\StringTok{ }\NormalTok{BASE[(BASE}\OperatorTok{$}\NormalTok{Fecha}\OperatorTok{>=}\StringTok{'2017-01-01'}\NormalTok{) }\OperatorTok{&}\StringTok{ }\NormalTok{(BASE}\OperatorTok{$}\NormalTok{Fecha}\OperatorTok{<=}\StringTok{'2017-06-30'}\NormalTok{) }\OperatorTok{&}\StringTok{ }\NormalTok{(BASE}\OperatorTok{$}\NormalTok{Moneda }\OperatorTok{==}\StringTok{ 'PESO'}\NormalTok{),] }\OperatorTok\StringTok{ }\KeywordTok{group_by}\NormalTok{(Empresa,Moneda) }\OperatorTok\StringTok{ }\KeywordTok{summarise}\NormalTok{(}\DataTypeTok{total =} \KeywordTok{sum}\NormalTok{(Ingreso))}
\end{Highlighting}
\end{Shaded}

\begin{verbatim}
## `summarise()` regrouping output by 'Empresa' (override with `.groups` argument)
\end{verbatim}

\begin{Shaded}
\begin{Highlighting}[]
\NormalTok{menor_ingreso}
\end{Highlighting}
\end{Shaded}

\begin{verbatim}
## # A tibble: 3 x 3
## # Groups:   Empresa [3]
##   Empresa   Moneda   total
##   <chr>     <chr>    <dbl>
## 1 Empesa B  PESO   174377.
## 2 Empresa A PESO    40276.
## 3 Empresa C PESO    76994.
\end{verbatim}

Vemos claramente que la empresa A fue la que menor ingreso generó

Ingreso para todos los trimestres del año 2017 en PESOS de cada una de
las empresas

\begin{Shaded}
\begin{Highlighting}[]
\NormalTok{BASE}\OperatorTok{$}\NormalTok{Trim <-}\StringTok{ }\KeywordTok{as.yearqtr}\NormalTok{(BASE}\OperatorTok{$}\NormalTok{Fecha, }\DataTypeTok{format =} \StringTok{"%Y-%m-%d"}\NormalTok{)}

\NormalTok{Trimestre <-}\StringTok{ }\NormalTok{BASE[BASE}\OperatorTok{$}\NormalTok{Moneda }\OperatorTok{==}\StringTok{ 'PESO'}\NormalTok{,] }\OperatorTok\StringTok{  }\KeywordTok{select}\NormalTok{(Trim,Empresa,Moneda,Ingreso) }\OperatorTok\StringTok{ }\KeywordTok{group_by}\NormalTok{(Trim,Empresa,Moneda) }\OperatorTok\StringTok{ }\KeywordTok{summarise}\NormalTok{(}\DataTypeTok{total =} \KeywordTok{sum}\NormalTok{(Ingreso))}
\end{Highlighting}
\end{Shaded}

\begin{verbatim}
## `summarise()` regrouping output by 'Trim', 'Empresa' (override with `.groups` argument)
\end{verbatim}

\begin{Shaded}
\begin{Highlighting}[]
\NormalTok{Trimestre}
\end{Highlighting}
\end{Shaded}

\begin{verbatim}
## # A tibble: 12 x 4
## # Groups:   Trim, Empresa [12]
##    Trim      Empresa   Moneda  total
##    <yearqtr> <chr>     <chr>   <dbl>
##  1 2017 Q1   Empesa B  PESO   89477.
##  2 2017 Q1   Empresa A PESO   20624.
##  3 2017 Q1   Empresa C PESO   39116.
##  4 2017 Q2   Empesa B  PESO   84900.
##  5 2017 Q2   Empresa A PESO   19652.
##  6 2017 Q2   Empresa C PESO   37878.
##  7 2017 Q3   Empesa B  PESO   66042.
##  8 2017 Q3   Empresa A PESO   12221.
##  9 2017 Q3   Empresa C PESO   25054.
## 10 2017 Q4   Empesa B  PESO   86824.
## 11 2017 Q4   Empresa A PESO   19171.
## 12 2017 Q4   Empresa C PESO   40690.
\end{verbatim}

Ingreso acumulado del año 2017 en PESOS desagregado por empresa y moneda

\begin{Shaded}
\begin{Highlighting}[]
\NormalTok{total_emp_moneda <-}\StringTok{ }\NormalTok{BASE  }\OperatorTok\StringTok{ }\KeywordTok{select}\NormalTok{(Empresa,Moneda,Ingreso) }\OperatorTok\StringTok{ }\KeywordTok{group_by}\NormalTok{(Empresa,Moneda )}\OperatorTok\StringTok{ }\KeywordTok{summarise}\NormalTok{(}\DataTypeTok{total =} \KeywordTok{sum}\NormalTok{(Ingreso))}
\end{Highlighting}
\end{Shaded}

\begin{verbatim}
## `summarise()` regrouping output by 'Empresa' (override with `.groups` argument)
\end{verbatim}

\begin{Shaded}
\begin{Highlighting}[]
\NormalTok{total_emp_moneda}
\end{Highlighting}
\end{Shaded}

\begin{verbatim}
## # A tibble: 9 x 3
## # Groups:   Empresa [3]
##   Empresa   Moneda      total
##   <chr>     <chr>       <dbl>
## 1 Empesa B  EURO        1.06 
## 2 Empesa B  PESO   327243.   
## 3 Empesa B  USD        15.7  
## 4 Empresa A EURO        0.575
## 5 Empresa A PESO    71668.   
## 6 Empresa A USD         9.86 
## 7 Empresa C EURO        0.349
## 8 Empresa C PESO   142738.   
## 9 Empresa C USD         6.22
\end{verbatim}


\end{document}
